\documentclass[11pt]{article}

    \usepackage[breakable]{tcolorbox}
    \usepackage{parskip} % Stop auto-indenting (to mimic markdown behaviour)
    

    % Basic figure setup, for now with no caption control since it's done
    % automatically by Pandoc (which extracts ![](path) syntax from Markdown).
    \usepackage{graphicx}
    % Maintain compatibility with old templates. Remove in nbconvert 6.0
    \let\Oldincludegraphics\includegraphics
    % Ensure that by default, figures have no caption (until we provide a
    % proper Figure object with a Caption API and a way to capture that
    % in the conversion process - todo).
    \usepackage{caption}
    \DeclareCaptionFormat{nocaption}{}
    \captionsetup{format=nocaption,aboveskip=0pt,belowskip=0pt}

    \usepackage{float}
    \floatplacement{figure}{H} % forces figures to be placed at the correct location
    \usepackage{xcolor} % Allow colors to be defined
    \usepackage{enumerate} % Needed for markdown enumerations to work
    \usepackage{geometry} % Used to adjust the document margins
    \usepackage{amsmath} % Equations
    \usepackage{amssymb} % Equations
    \usepackage{textcomp} % defines textquotesingle
    % Hack from http://tex.stackexchange.com/a/47451/13684:
    \AtBeginDocument{%
        \def\PYZsq{\textquotesingle}% Upright quotes in Pygmentized code
    }
    \usepackage{upquote} % Upright quotes for verbatim code
    \usepackage{eurosym} % defines \euro

    \usepackage{iftex}
    \ifPDFTeX
        \usepackage[T1]{fontenc}
        \IfFileExists{alphabeta.sty}{
              \usepackage{alphabeta}
          }{
              \usepackage[mathletters]{ucs}
              \usepackage[utf8x]{inputenc}
          }
    \else
        \usepackage{fontspec}
        \usepackage{unicode-math}
    \fi

    \usepackage{fancyvrb} % verbatim replacement that allows latex
    \usepackage{grffile} % extends the file name processing of package graphics
                         % to support a larger range
    \makeatletter % fix for old versions of grffile with XeLaTeX
    \@ifpackagelater{grffile}{2019/11/01}
    {
      % Do nothing on new versions
    }
    {
      \def\Gread@@xetex#1{%
        \IfFileExists{"\Gin@base".bb}%
        {\Gread@eps{\Gin@base.bb}}%
        {\Gread@@xetex@aux#1}%
      }
    }
    \makeatother
    \usepackage[Export]{adjustbox} % Used to constrain images to a maximum size
    \adjustboxset{max size={0.9\linewidth}{0.9\paperheight}}

    % The hyperref package gives us a pdf with properly built
    % internal navigation ('pdf bookmarks' for the table of contents,
    % internal cross-reference links, web links for URLs, etc.)
    \usepackage{hyperref}
    % The default LaTeX title has an obnoxious amount of whitespace. By default,
    % titling removes some of it. It also provides customization options.
    \usepackage{titling}
    \usepackage{longtable} % longtable support required by pandoc >1.10
    \usepackage{booktabs}  % table support for pandoc > 1.12.2
    \usepackage{array}     % table support for pandoc >= 2.11.3
    \usepackage{calc}      % table minipage width calculation for pandoc >= 2.11.1
    \usepackage[inline]{enumitem} % IRkernel/repr support (it uses the enumerate* environment)
    \usepackage[normalem]{ulem} % ulem is needed to support strikethroughs (\sout)
                                % normalem makes italics be italics, not underlines
    \usepackage{mathrsfs}
    

    
    % Colors for the hyperref package
    \definecolor{urlcolor}{rgb}{0,.145,.698}
    \definecolor{linkcolor}{rgb}{.71,0.21,0.01}
    \definecolor{citecolor}{rgb}{.12,.54,.11}

    % ANSI colors
    \definecolor{ansi-black}{HTML}{3E424D}
    \definecolor{ansi-black-intense}{HTML}{282C36}
    \definecolor{ansi-red}{HTML}{E75C58}
    \definecolor{ansi-red-intense}{HTML}{B22B31}
    \definecolor{ansi-green}{HTML}{00A250}
    \definecolor{ansi-green-intense}{HTML}{007427}
    \definecolor{ansi-yellow}{HTML}{DDB62B}
    \definecolor{ansi-yellow-intense}{HTML}{B27D12}
    \definecolor{ansi-blue}{HTML}{208FFB}
    \definecolor{ansi-blue-intense}{HTML}{0065CA}
    \definecolor{ansi-magenta}{HTML}{D160C4}
    \definecolor{ansi-magenta-intense}{HTML}{A03196}
    \definecolor{ansi-cyan}{HTML}{60C6C8}
    \definecolor{ansi-cyan-intense}{HTML}{258F8F}
    \definecolor{ansi-white}{HTML}{C5C1B4}
    \definecolor{ansi-white-intense}{HTML}{A1A6B2}
    \definecolor{ansi-default-inverse-fg}{HTML}{FFFFFF}
    \definecolor{ansi-default-inverse-bg}{HTML}{000000}

    % common color for the border for error outputs.
    \definecolor{outerrorbackground}{HTML}{FFDFDF}

    % commands and environments needed by pandoc snippets
    % extracted from the output of `pandoc -s`
    \providecommand{\tightlist}{%
      \setlength{\itemsep}{0pt}\setlength{\parskip}{0pt}}
    \DefineVerbatimEnvironment{Highlighting}{Verbatim}{commandchars=\\\{\}}
    % Add ',fontsize=\small' for more characters per line
    \newenvironment{Shaded}{}{}
    \newcommand{\KeywordTok}[1]{\textcolor[rgb]{0.00,0.44,0.13}{\textbf{{#1}}}}
    \newcommand{\DataTypeTok}[1]{\textcolor[rgb]{0.56,0.13,0.00}{{#1}}}
    \newcommand{\DecValTok}[1]{\textcolor[rgb]{0.25,0.63,0.44}{{#1}}}
    \newcommand{\BaseNTok}[1]{\textcolor[rgb]{0.25,0.63,0.44}{{#1}}}
    \newcommand{\FloatTok}[1]{\textcolor[rgb]{0.25,0.63,0.44}{{#1}}}
    \newcommand{\CharTok}[1]{\textcolor[rgb]{0.25,0.44,0.63}{{#1}}}
    \newcommand{\StringTok}[1]{\textcolor[rgb]{0.25,0.44,0.63}{{#1}}}
    \newcommand{\CommentTok}[1]{\textcolor[rgb]{0.38,0.63,0.69}{\textit{{#1}}}}
    \newcommand{\OtherTok}[1]{\textcolor[rgb]{0.00,0.44,0.13}{{#1}}}
    \newcommand{\AlertTok}[1]{\textcolor[rgb]{1.00,0.00,0.00}{\textbf{{#1}}}}
    \newcommand{\FunctionTok}[1]{\textcolor[rgb]{0.02,0.16,0.49}{{#1}}}
    \newcommand{\RegionMarkerTok}[1]{{#1}}
    \newcommand{\ErrorTok}[1]{\textcolor[rgb]{1.00,0.00,0.00}{\textbf{{#1}}}}
    \newcommand{\NormalTok}[1]{{#1}}

    % Additional commands for more recent versions of Pandoc
    \newcommand{\ConstantTok}[1]{\textcolor[rgb]{0.53,0.00,0.00}{{#1}}}
    \newcommand{\SpecialCharTok}[1]{\textcolor[rgb]{0.25,0.44,0.63}{{#1}}}
    \newcommand{\VerbatimStringTok}[1]{\textcolor[rgb]{0.25,0.44,0.63}{{#1}}}
    \newcommand{\SpecialStringTok}[1]{\textcolor[rgb]{0.73,0.40,0.53}{{#1}}}
    \newcommand{\ImportTok}[1]{{#1}}
    \newcommand{\DocumentationTok}[1]{\textcolor[rgb]{0.73,0.13,0.13}{\textit{{#1}}}}
    \newcommand{\AnnotationTok}[1]{\textcolor[rgb]{0.38,0.63,0.69}{\textbf{\textit{{#1}}}}}
    \newcommand{\CommentVarTok}[1]{\textcolor[rgb]{0.38,0.63,0.69}{\textbf{\textit{{#1}}}}}
    \newcommand{\VariableTok}[1]{\textcolor[rgb]{0.10,0.09,0.49}{{#1}}}
    \newcommand{\ControlFlowTok}[1]{\textcolor[rgb]{0.00,0.44,0.13}{\textbf{{#1}}}}
    \newcommand{\OperatorTok}[1]{\textcolor[rgb]{0.40,0.40,0.40}{{#1}}}
    \newcommand{\BuiltInTok}[1]{{#1}}
    \newcommand{\ExtensionTok}[1]{{#1}}
    \newcommand{\PreprocessorTok}[1]{\textcolor[rgb]{0.74,0.48,0.00}{{#1}}}
    \newcommand{\AttributeTok}[1]{\textcolor[rgb]{0.49,0.56,0.16}{{#1}}}
    \newcommand{\InformationTok}[1]{\textcolor[rgb]{0.38,0.63,0.69}{\textbf{\textit{{#1}}}}}
    \newcommand{\WarningTok}[1]{\textcolor[rgb]{0.38,0.63,0.69}{\textbf{\textit{{#1}}}}}


    % Define a nice break command that doesn't care if a line doesn't already
    % exist.
    \def\br{\hspace*{\fill} \\* }
    % Math Jax compatibility definitions
    \def\gt{>}
    \def\lt{<}
    \let\Oldtex\TeX
    \let\Oldlatex\LaTeX
    \renewcommand{\TeX}{\textrm{\Oldtex}}
    \renewcommand{\LaTeX}{\textrm{\Oldlatex}}
    % Document parameters
    % Document title
    \title{CreateAnMXNetImageClassifier}
    
    
    
    
    
% Pygments definitions
\makeatletter
\def\PY@reset{\let\PY@it=\relax \let\PY@bf=\relax%
    \let\PY@ul=\relax \let\PY@tc=\relax%
    \let\PY@bc=\relax \let\PY@ff=\relax}
\def\PY@tok#1{\csname PY@tok@#1\endcsname}
\def\PY@toks#1+{\ifx\relax#1\empty\else%
    \PY@tok{#1}\expandafter\PY@toks\fi}
\def\PY@do#1{\PY@bc{\PY@tc{\PY@ul{%
    \PY@it{\PY@bf{\PY@ff{#1}}}}}}}
\def\PY#1#2{\PY@reset\PY@toks#1+\relax+\PY@do{#2}}

\@namedef{PY@tok@w}{\def\PY@tc##1{\textcolor[rgb]{0.73,0.73,0.73}{##1}}}
\@namedef{PY@tok@c}{\let\PY@it=\textit\def\PY@tc##1{\textcolor[rgb]{0.24,0.48,0.48}{##1}}}
\@namedef{PY@tok@cp}{\def\PY@tc##1{\textcolor[rgb]{0.61,0.40,0.00}{##1}}}
\@namedef{PY@tok@k}{\let\PY@bf=\textbf\def\PY@tc##1{\textcolor[rgb]{0.00,0.50,0.00}{##1}}}
\@namedef{PY@tok@kp}{\def\PY@tc##1{\textcolor[rgb]{0.00,0.50,0.00}{##1}}}
\@namedef{PY@tok@kt}{\def\PY@tc##1{\textcolor[rgb]{0.69,0.00,0.25}{##1}}}
\@namedef{PY@tok@o}{\def\PY@tc##1{\textcolor[rgb]{0.40,0.40,0.40}{##1}}}
\@namedef{PY@tok@ow}{\let\PY@bf=\textbf\def\PY@tc##1{\textcolor[rgb]{0.67,0.13,1.00}{##1}}}
\@namedef{PY@tok@nb}{\def\PY@tc##1{\textcolor[rgb]{0.00,0.50,0.00}{##1}}}
\@namedef{PY@tok@nf}{\def\PY@tc##1{\textcolor[rgb]{0.00,0.00,1.00}{##1}}}
\@namedef{PY@tok@nc}{\let\PY@bf=\textbf\def\PY@tc##1{\textcolor[rgb]{0.00,0.00,1.00}{##1}}}
\@namedef{PY@tok@nn}{\let\PY@bf=\textbf\def\PY@tc##1{\textcolor[rgb]{0.00,0.00,1.00}{##1}}}
\@namedef{PY@tok@ne}{\let\PY@bf=\textbf\def\PY@tc##1{\textcolor[rgb]{0.80,0.25,0.22}{##1}}}
\@namedef{PY@tok@nv}{\def\PY@tc##1{\textcolor[rgb]{0.10,0.09,0.49}{##1}}}
\@namedef{PY@tok@no}{\def\PY@tc##1{\textcolor[rgb]{0.53,0.00,0.00}{##1}}}
\@namedef{PY@tok@nl}{\def\PY@tc##1{\textcolor[rgb]{0.46,0.46,0.00}{##1}}}
\@namedef{PY@tok@ni}{\let\PY@bf=\textbf\def\PY@tc##1{\textcolor[rgb]{0.44,0.44,0.44}{##1}}}
\@namedef{PY@tok@na}{\def\PY@tc##1{\textcolor[rgb]{0.41,0.47,0.13}{##1}}}
\@namedef{PY@tok@nt}{\let\PY@bf=\textbf\def\PY@tc##1{\textcolor[rgb]{0.00,0.50,0.00}{##1}}}
\@namedef{PY@tok@nd}{\def\PY@tc##1{\textcolor[rgb]{0.67,0.13,1.00}{##1}}}
\@namedef{PY@tok@s}{\def\PY@tc##1{\textcolor[rgb]{0.73,0.13,0.13}{##1}}}
\@namedef{PY@tok@sd}{\let\PY@it=\textit\def\PY@tc##1{\textcolor[rgb]{0.73,0.13,0.13}{##1}}}
\@namedef{PY@tok@si}{\let\PY@bf=\textbf\def\PY@tc##1{\textcolor[rgb]{0.64,0.35,0.47}{##1}}}
\@namedef{PY@tok@se}{\let\PY@bf=\textbf\def\PY@tc##1{\textcolor[rgb]{0.67,0.36,0.12}{##1}}}
\@namedef{PY@tok@sr}{\def\PY@tc##1{\textcolor[rgb]{0.64,0.35,0.47}{##1}}}
\@namedef{PY@tok@ss}{\def\PY@tc##1{\textcolor[rgb]{0.10,0.09,0.49}{##1}}}
\@namedef{PY@tok@sx}{\def\PY@tc##1{\textcolor[rgb]{0.00,0.50,0.00}{##1}}}
\@namedef{PY@tok@m}{\def\PY@tc##1{\textcolor[rgb]{0.40,0.40,0.40}{##1}}}
\@namedef{PY@tok@gh}{\let\PY@bf=\textbf\def\PY@tc##1{\textcolor[rgb]{0.00,0.00,0.50}{##1}}}
\@namedef{PY@tok@gu}{\let\PY@bf=\textbf\def\PY@tc##1{\textcolor[rgb]{0.50,0.00,0.50}{##1}}}
\@namedef{PY@tok@gd}{\def\PY@tc##1{\textcolor[rgb]{0.63,0.00,0.00}{##1}}}
\@namedef{PY@tok@gi}{\def\PY@tc##1{\textcolor[rgb]{0.00,0.52,0.00}{##1}}}
\@namedef{PY@tok@gr}{\def\PY@tc##1{\textcolor[rgb]{0.89,0.00,0.00}{##1}}}
\@namedef{PY@tok@ge}{\let\PY@it=\textit}
\@namedef{PY@tok@gs}{\let\PY@bf=\textbf}
\@namedef{PY@tok@ges}{\let\PY@bf=\textbf\let\PY@it=\textit}
\@namedef{PY@tok@gp}{\let\PY@bf=\textbf\def\PY@tc##1{\textcolor[rgb]{0.00,0.00,0.50}{##1}}}
\@namedef{PY@tok@go}{\def\PY@tc##1{\textcolor[rgb]{0.44,0.44,0.44}{##1}}}
\@namedef{PY@tok@gt}{\def\PY@tc##1{\textcolor[rgb]{0.00,0.27,0.87}{##1}}}
\@namedef{PY@tok@err}{\def\PY@bc##1{{\setlength{\fboxsep}{\string -\fboxrule}\fcolorbox[rgb]{1.00,0.00,0.00}{1,1,1}{\strut ##1}}}}
\@namedef{PY@tok@kc}{\let\PY@bf=\textbf\def\PY@tc##1{\textcolor[rgb]{0.00,0.50,0.00}{##1}}}
\@namedef{PY@tok@kd}{\let\PY@bf=\textbf\def\PY@tc##1{\textcolor[rgb]{0.00,0.50,0.00}{##1}}}
\@namedef{PY@tok@kn}{\let\PY@bf=\textbf\def\PY@tc##1{\textcolor[rgb]{0.00,0.50,0.00}{##1}}}
\@namedef{PY@tok@kr}{\let\PY@bf=\textbf\def\PY@tc##1{\textcolor[rgb]{0.00,0.50,0.00}{##1}}}
\@namedef{PY@tok@bp}{\def\PY@tc##1{\textcolor[rgb]{0.00,0.50,0.00}{##1}}}
\@namedef{PY@tok@fm}{\def\PY@tc##1{\textcolor[rgb]{0.00,0.00,1.00}{##1}}}
\@namedef{PY@tok@vc}{\def\PY@tc##1{\textcolor[rgb]{0.10,0.09,0.49}{##1}}}
\@namedef{PY@tok@vg}{\def\PY@tc##1{\textcolor[rgb]{0.10,0.09,0.49}{##1}}}
\@namedef{PY@tok@vi}{\def\PY@tc##1{\textcolor[rgb]{0.10,0.09,0.49}{##1}}}
\@namedef{PY@tok@vm}{\def\PY@tc##1{\textcolor[rgb]{0.10,0.09,0.49}{##1}}}
\@namedef{PY@tok@sa}{\def\PY@tc##1{\textcolor[rgb]{0.73,0.13,0.13}{##1}}}
\@namedef{PY@tok@sb}{\def\PY@tc##1{\textcolor[rgb]{0.73,0.13,0.13}{##1}}}
\@namedef{PY@tok@sc}{\def\PY@tc##1{\textcolor[rgb]{0.73,0.13,0.13}{##1}}}
\@namedef{PY@tok@dl}{\def\PY@tc##1{\textcolor[rgb]{0.73,0.13,0.13}{##1}}}
\@namedef{PY@tok@s2}{\def\PY@tc##1{\textcolor[rgb]{0.73,0.13,0.13}{##1}}}
\@namedef{PY@tok@sh}{\def\PY@tc##1{\textcolor[rgb]{0.73,0.13,0.13}{##1}}}
\@namedef{PY@tok@s1}{\def\PY@tc##1{\textcolor[rgb]{0.73,0.13,0.13}{##1}}}
\@namedef{PY@tok@mb}{\def\PY@tc##1{\textcolor[rgb]{0.40,0.40,0.40}{##1}}}
\@namedef{PY@tok@mf}{\def\PY@tc##1{\textcolor[rgb]{0.40,0.40,0.40}{##1}}}
\@namedef{PY@tok@mh}{\def\PY@tc##1{\textcolor[rgb]{0.40,0.40,0.40}{##1}}}
\@namedef{PY@tok@mi}{\def\PY@tc##1{\textcolor[rgb]{0.40,0.40,0.40}{##1}}}
\@namedef{PY@tok@il}{\def\PY@tc##1{\textcolor[rgb]{0.40,0.40,0.40}{##1}}}
\@namedef{PY@tok@mo}{\def\PY@tc##1{\textcolor[rgb]{0.40,0.40,0.40}{##1}}}
\@namedef{PY@tok@ch}{\let\PY@it=\textit\def\PY@tc##1{\textcolor[rgb]{0.24,0.48,0.48}{##1}}}
\@namedef{PY@tok@cm}{\let\PY@it=\textit\def\PY@tc##1{\textcolor[rgb]{0.24,0.48,0.48}{##1}}}
\@namedef{PY@tok@cpf}{\let\PY@it=\textit\def\PY@tc##1{\textcolor[rgb]{0.24,0.48,0.48}{##1}}}
\@namedef{PY@tok@c1}{\let\PY@it=\textit\def\PY@tc##1{\textcolor[rgb]{0.24,0.48,0.48}{##1}}}
\@namedef{PY@tok@cs}{\let\PY@it=\textit\def\PY@tc##1{\textcolor[rgb]{0.24,0.48,0.48}{##1}}}

\def\PYZbs{\char`\\}
\def\PYZus{\char`\_}
\def\PYZob{\char`\{}
\def\PYZcb{\char`\}}
\def\PYZca{\char`\^}
\def\PYZam{\char`\&}
\def\PYZlt{\char`\<}
\def\PYZgt{\char`\>}
\def\PYZsh{\char`\#}
\def\PYZpc{\char`\%}
\def\PYZdl{\char`\$}
\def\PYZhy{\char`\-}
\def\PYZsq{\char`\'}
\def\PYZdq{\char`\"}
\def\PYZti{\char`\~}
% for compatibility with earlier versions
\def\PYZat{@}
\def\PYZlb{[}
\def\PYZrb{]}
\makeatother


    % For linebreaks inside Verbatim environment from package fancyvrb.
    \makeatletter
        \newbox\Wrappedcontinuationbox
        \newbox\Wrappedvisiblespacebox
        \newcommand*\Wrappedvisiblespace {\textcolor{red}{\textvisiblespace}}
        \newcommand*\Wrappedcontinuationsymbol {\textcolor{red}{\llap{\tiny$\m@th\hookrightarrow$}}}
        \newcommand*\Wrappedcontinuationindent {3ex }
        \newcommand*\Wrappedafterbreak {\kern\Wrappedcontinuationindent\copy\Wrappedcontinuationbox}
        % Take advantage of the already applied Pygments mark-up to insert
        % potential linebreaks for TeX processing.
        %        {, <, #, %, $, ' and ": go to next line.
        %        _, }, ^, &, >, - and ~: stay at end of broken line.
        % Use of \textquotesingle for straight quote.
        \newcommand*\Wrappedbreaksatspecials {%
            \def\PYGZus{\discretionary{\char`\_}{\Wrappedafterbreak}{\char`\_}}%
            \def\PYGZob{\discretionary{}{\Wrappedafterbreak\char`\{}{\char`\{}}%
            \def\PYGZcb{\discretionary{\char`\}}{\Wrappedafterbreak}{\char`\}}}%
            \def\PYGZca{\discretionary{\char`\^}{\Wrappedafterbreak}{\char`\^}}%
            \def\PYGZam{\discretionary{\char`\&}{\Wrappedafterbreak}{\char`\&}}%
            \def\PYGZlt{\discretionary{}{\Wrappedafterbreak\char`\<}{\char`\<}}%
            \def\PYGZgt{\discretionary{\char`\>}{\Wrappedafterbreak}{\char`\>}}%
            \def\PYGZsh{\discretionary{}{\Wrappedafterbreak\char`\#}{\char`\#}}%
            \def\PYGZpc{\discretionary{}{\Wrappedafterbreak\char`\%}{\char`\%}}%
            \def\PYGZdl{\discretionary{}{\Wrappedafterbreak\char`\$}{\char`\$}}%
            \def\PYGZhy{\discretionary{\char`\-}{\Wrappedafterbreak}{\char`\-}}%
            \def\PYGZsq{\discretionary{}{\Wrappedafterbreak\textquotesingle}{\textquotesingle}}%
            \def\PYGZdq{\discretionary{}{\Wrappedafterbreak\char`\"}{\char`\"}}%
            \def\PYGZti{\discretionary{\char`\~}{\Wrappedafterbreak}{\char`\~}}%
        }
        % Some characters . , ; ? ! / are not pygmentized.
        % This macro makes them "active" and they will insert potential linebreaks
        \newcommand*\Wrappedbreaksatpunct {%
            \lccode`\~`\.\lowercase{\def~}{\discretionary{\hbox{\char`\.}}{\Wrappedafterbreak}{\hbox{\char`\.}}}%
            \lccode`\~`\,\lowercase{\def~}{\discretionary{\hbox{\char`\,}}{\Wrappedafterbreak}{\hbox{\char`\,}}}%
            \lccode`\~`\;\lowercase{\def~}{\discretionary{\hbox{\char`\;}}{\Wrappedafterbreak}{\hbox{\char`\;}}}%
            \lccode`\~`\:\lowercase{\def~}{\discretionary{\hbox{\char`\:}}{\Wrappedafterbreak}{\hbox{\char`\:}}}%
            \lccode`\~`\?\lowercase{\def~}{\discretionary{\hbox{\char`\?}}{\Wrappedafterbreak}{\hbox{\char`\?}}}%
            \lccode`\~`\!\lowercase{\def~}{\discretionary{\hbox{\char`\!}}{\Wrappedafterbreak}{\hbox{\char`\!}}}%
            \lccode`\~`\/\lowercase{\def~}{\discretionary{\hbox{\char`\/}}{\Wrappedafterbreak}{\hbox{\char`\/}}}%
            \catcode`\.\active
            \catcode`\,\active
            \catcode`\;\active
            \catcode`\:\active
            \catcode`\?\active
            \catcode`\!\active
            \catcode`\/\active
            \lccode`\~`\~
        }
    \makeatother

    \let\OriginalVerbatim=\Verbatim
    \makeatletter
    \renewcommand{\Verbatim}[1][1]{%
        %\parskip\z@skip
        \sbox\Wrappedcontinuationbox {\Wrappedcontinuationsymbol}%
        \sbox\Wrappedvisiblespacebox {\FV@SetupFont\Wrappedvisiblespace}%
        \def\FancyVerbFormatLine ##1{\hsize\linewidth
            \vtop{\raggedright\hyphenpenalty\z@\exhyphenpenalty\z@
                \doublehyphendemerits\z@\finalhyphendemerits\z@
                \strut ##1\strut}%
        }%
        % If the linebreak is at a space, the latter will be displayed as visible
        % space at end of first line, and a continuation symbol starts next line.
        % Stretch/shrink are however usually zero for typewriter font.
        \def\FV@Space {%
            \nobreak\hskip\z@ plus\fontdimen3\font minus\fontdimen4\font
            \discretionary{\copy\Wrappedvisiblespacebox}{\Wrappedafterbreak}
            {\kern\fontdimen2\font}%
        }%

        % Allow breaks at special characters using \PYG... macros.
        \Wrappedbreaksatspecials
        % Breaks at punctuation characters . , ; ? ! and / need catcode=\active
        \OriginalVerbatim[#1,codes*=\Wrappedbreaksatpunct]%
    }
    \makeatother

    % Exact colors from NB
    \definecolor{incolor}{HTML}{303F9F}
    \definecolor{outcolor}{HTML}{D84315}
    \definecolor{cellborder}{HTML}{CFCFCF}
    \definecolor{cellbackground}{HTML}{F7F7F7}

    % prompt
    \makeatletter
    \newcommand{\boxspacing}{\kern\kvtcb@left@rule\kern\kvtcb@boxsep}
    \makeatother
    \newcommand{\prompt}[4]{
        {\ttfamily\llap{{\color{#2}[#3]:\hspace{3pt}#4}}\vspace{-\baselineskip}}
    }
    

    
    % Prevent overflowing lines due to hard-to-break entities
    \sloppy
    % Setup hyperref package
    \hypersetup{
      breaklinks=true,  % so long urls are correctly broken across lines
      colorlinks=true,
      urlcolor=urlcolor,
      linkcolor=linkcolor,
      citecolor=citecolor,
      }
    % Slightly bigger margins than the latex defaults
    
    \geometry{verbose,tmargin=1in,bmargin=1in,lmargin=1in,rmargin=1in}
    
    

\begin{document}
    
    \maketitle
    
    

    
    \begin{figure}
\centering
\includegraphics{acg_logo.png}
\caption{A Cloud Guru}
\end{figure}

    MXNet Image Classification

    \begin{figure}
\centering
\includegraphics{./lego.jpg}
\caption{Sorting Lego bricks}
\end{figure}

    Lego Brick Sorting

    \section{Frameworks}\label{frameworks}

For this lab, we will be using Apache MXNet to build and train a model
to classify images, and specifically take advantage of the Gluon API
provided with MXNet to make that process really easy.

\subsection{MXNet}\label{mxnet}

A flexible and efficient library for deep learning.

\begin{itemize}
\tightlist
\item
  MXNet provides optimized numerical computation for GPUs and
  distributed ecosystems, from the comfort of high-level environments
  like Python and R.
\item
  MXNet automates common workflows, so standard neural networks can be
  expressed concisely in just a few lines of code.
\end{itemize}

\emph{(Source: https://mxnet.apache.org/)}

\subsection{Gluon}\label{gluon}

Based on the the Gluon API specification, the Gluon library in Apache
MXNet provides a clear, concise, and simple API for deep learning. It
provides basic building blocks for neural networks that make it easy to
prototype, build, and train deep learning models without sacrificing
training speed. Install a recent version of MXNet to get access to
Gluon.

\emph{(Source:
\href{https://aws.amazon.com/blogs/machine-learning/introducing-gluon-an-easy-to-use-programming-interface-for-flexible-deep-learning/}{AWS
Blog})}

\section{Scenario}\label{scenario}

We have bricks. Lots of bricks. LEGO bricks, that is. And we need to
sort them.

We also have a collection of photos of various LEGO bricks from
different angles. We have 600 photos (which probably took us more time
to collect than sorting the current bricks) and they are all labeled
with the brick type.

Each photo has been processed. This involved increasing the contrast,
sharpening, removing the color, inverting the colors, and reducing the
size.

\begin{longtable}[]{@{}
  >{\raggedright\arraybackslash}p{(\columnwidth - 2\tabcolsep) * \real{0.5000}}
  >{\raggedright\arraybackslash}p{(\columnwidth - 2\tabcolsep) * \real{0.5000}}@{}}
\toprule\noalign{}
\begin{minipage}[b]{\linewidth}\raggedright
\includegraphics{./sample-before.png}
\end{minipage} & \begin{minipage}[b]{\linewidth}\raggedright
\includegraphics{./sample-after.png}
\end{minipage} \\
\midrule\noalign{}
\endhead
\bottomrule\noalign{}
\endlastfoot
Sample before processing & Sample after processing \\
\end{longtable}

In addition to this, we stored all the images into data arrays for
easier loading into the notebook. These are stored in the
\texttt{lego-simple-mx-train} and \texttt{lego-simple-mx-test} files.

We need to create a simple, deep learning, neural network classifier
model. We will train the model using the photo data and see if it
correctly predicts the type of a brick from a supplied test image.

    \section{How to Use This Lab}\label{how-to-use-this-lab}

All of the code is provided for you in this lab as our solution to the
tasks presented. You could simply execute the notebook to get a result,
but that's not really very hands-on and it won't teach you anything but
how to execute cells in a Jupyter notebook. To get the most from this
lab, you should understand what the code in each cell is trying to
accomplish, and then take the time to experiment: make changes, break
it, fix it, and learn! You can always pull the code down again to get a
clean copy.

    \section{1) Preparing our Environment}\label{preparing-our-environment}

    First, we'll need to install the \texttt{mxnet} package to run
everything. Since it's not included in the \texttt{conda\_python} image
by default, we'll need to use \texttt{pip} to install it.

    \begin{tcolorbox}[breakable, size=fbox, boxrule=1pt, pad at break*=1mm,colback=cellbackground, colframe=cellborder]
\prompt{In}{incolor}{1}{\boxspacing}
\begin{Verbatim}[commandchars=\\\{\}]
\PY{o}{\PYZpc{}}\PY{k}{pip} install mxnet
\end{Verbatim}
\end{tcolorbox}

    \begin{Verbatim}[commandchars=\\\{\}]
Requirement already satisfied: mxnet in
/home/ec2-user/anaconda3/envs/python3/lib/python3.10/site-packages (1.9.1)
Requirement already satisfied: numpy<2.0.0,>1.16.0 in
/home/ec2-user/anaconda3/envs/python3/lib/python3.10/site-packages (from mxnet)
(1.22.4)
Requirement already satisfied: requests<3,>=2.20.0 in
/home/ec2-user/anaconda3/envs/python3/lib/python3.10/site-packages (from mxnet)
(2.31.0)
Requirement already satisfied: graphviz<0.9.0,>=0.8.1 in
/home/ec2-user/anaconda3/envs/python3/lib/python3.10/site-packages (from mxnet)
(0.8.4)
Requirement already satisfied: charset-normalizer<4,>=2 in
/home/ec2-user/anaconda3/envs/python3/lib/python3.10/site-packages (from
requests<3,>=2.20.0->mxnet) (3.3.2)
Requirement already satisfied: idna<4,>=2.5 in
/home/ec2-user/anaconda3/envs/python3/lib/python3.10/site-packages (from
requests<3,>=2.20.0->mxnet) (3.6)
Requirement already satisfied: urllib3<3,>=1.21.1 in
/home/ec2-user/anaconda3/envs/python3/lib/python3.10/site-packages (from
requests<3,>=2.20.0->mxnet) (2.2.1)
Requirement already satisfied: certifi>=2017.4.17 in
/home/ec2-user/anaconda3/envs/python3/lib/python3.10/site-packages (from
requests<3,>=2.20.0->mxnet) (2024.2.2)
Note: you may need to restart the kernel to use updated packages.
    \end{Verbatim}

    Now that MXNet has been installed, we can import all of the modules
we'll need to build and run our image classification model.

    \begin{tcolorbox}[breakable, size=fbox, boxrule=1pt, pad at break*=1mm,colback=cellbackground, colframe=cellborder]
\prompt{In}{incolor}{5}{\boxspacing}
\begin{Verbatim}[commandchars=\\\{\}]
\PY{k+kn}{import} \PY{n+nn}{mxnet} \PY{k}{as} \PY{n+nn}{mx}
\PY{k+kn}{from} \PY{n+nn}{mxnet}\PY{n+nn}{.}\PY{n+nn}{gluon}\PY{n+nn}{.}\PY{n+nn}{data}\PY{n+nn}{.}\PY{n+nn}{vision} \PY{k+kn}{import} \PY{n}{transforms}

\PY{k+kn}{from} \PY{n+nn}{mxnet} \PY{k+kn}{import} \PY{n}{nd}\PY{p}{,} \PY{n}{gluon}\PY{p}{,} \PY{n}{autograd}
\PY{k+kn}{from} \PY{n+nn}{mxnet}\PY{n+nn}{.}\PY{n+nn}{gluon} \PY{k+kn}{import} \PY{n}{nn}

\PY{k+kn}{import} \PY{n+nn}{pickle}

\PY{k+kn}{import} \PY{n+nn}{numpy} \PY{k}{as} \PY{n+nn}{np}
\PY{k+kn}{import} \PY{n+nn}{matplotlib}\PY{n+nn}{.}\PY{n+nn}{pyplot} \PY{k}{as} \PY{n+nn}{plt}
\end{Verbatim}
\end{tcolorbox}

    Since this is part of a learning environment, we'll also include a fixed
seed. This means your results should follow what you experience in the
hands-on lab, although you can also change this, or comment it out
entirely for different results.

And of course, there's only one seed that which would be appropriate to
help us figure out life, the universe, and everything

    \begin{tcolorbox}[breakable, size=fbox, boxrule=1pt, pad at break*=1mm,colback=cellbackground, colframe=cellborder]
\prompt{In}{incolor}{10}{\boxspacing}
\begin{Verbatim}[commandchars=\\\{\}]
\PY{n}{mx}\PY{o}{.}\PY{n}{random}\PY{o}{.}\PY{n}{seed}\PY{p}{(}\PY{l+m+mi}{42}\PY{p}{)}
\end{Verbatim}
\end{tcolorbox}

    Lastly, we need to set the processor type MXNet will use. This Juypter
Notebook server doesn't have access to GPU's and we let MXNet discover
that here. That will be okay for what we are doing.

    \begin{tcolorbox}[breakable, size=fbox, boxrule=1pt, pad at break*=1mm,colback=cellbackground, colframe=cellborder]
\prompt{In}{incolor}{11}{\boxspacing}
\begin{Verbatim}[commandchars=\\\{\}]
\PY{n}{ctx} \PY{o}{=} \PY{n}{mx}\PY{o}{.}\PY{n}{gpu}\PY{p}{(}\PY{l+m+mi}{0}\PY{p}{)} \PY{k}{if} \PY{n}{mx}\PY{o}{.}\PY{n}{context}\PY{o}{.}\PY{n}{num\PYZus{}gpus}\PY{p}{(}\PY{p}{)} \PY{o}{\PYZgt{}} \PY{l+m+mi}{0} \PY{k}{else} \PY{n}{mx}\PY{o}{.}\PY{n}{cpu}\PY{p}{(}\PY{l+m+mi}{0}\PY{p}{)}
\end{Verbatim}
\end{tcolorbox}

    \section{2) Load the Data}\label{load-the-data}

    We have a dataset created from a set of photos of LEGO bricks. In total
we have 2 sets of data saved to files as NDArrays.

\begin{enumerate}
\def\labelenumi{\arabic{enumi}.}
\tightlist
\item
  \textbf{lego-simple-mx-train} - \emph{Training images and labels
  combined, around 80\% of the data collected.}
\item
  \textbf{lego-simple-mx-test} - \emph{Testing or validation images and
  labels combined, around 20\% of the data collected.}
\end{enumerate}

First we load the data into runtime arrays. \texttt{Pickle} has been
used to create these object files, so we use \texttt{pickle} to load
them as well.

    \begin{tcolorbox}[breakable, size=fbox, boxrule=1pt, pad at break*=1mm,colback=cellbackground, colframe=cellborder]
\prompt{In}{incolor}{23}{\boxspacing}
\begin{Verbatim}[commandchars=\\\{\}]
\PY{n}{train\PYZus{}fh} \PY{o}{=} \PY{n+nb}{open}\PY{p}{(}\PY{l+s+s1}{\PYZsq{}}\PY{l+s+s1}{lego\PYZhy{}simple\PYZhy{}mx\PYZhy{}train}\PY{l+s+s1}{\PYZsq{}}\PY{p}{,} \PY{l+s+s1}{\PYZsq{}}\PY{l+s+s1}{rb}\PY{l+s+s1}{\PYZsq{}}\PY{p}{)}
\PY{n}{test\PYZus{}fh} \PY{o}{=} \PY{n+nb}{open}\PY{p}{(}\PY{l+s+s1}{\PYZsq{}}\PY{l+s+s1}{lego\PYZhy{}simple\PYZhy{}mx\PYZhy{}test}\PY{l+s+s1}{\PYZsq{}}\PY{p}{,} \PY{l+s+s1}{\PYZsq{}}\PY{l+s+s1}{rb}\PY{l+s+s1}{\PYZsq{}}\PY{p}{)}

\PY{n}{train\PYZus{}data} \PY{o}{=} \PY{n}{pickle}\PY{o}{.}\PY{n}{load}\PY{p}{(}\PY{n}{train\PYZus{}fh}\PY{p}{)}
\PY{n}{test\PYZus{}data} \PY{o}{=} \PY{n}{pickle}\PY{o}{.}\PY{n}{load}\PY{p}{(}\PY{n}{test\PYZus{}fh}\PY{p}{)}
\end{Verbatim}
\end{tcolorbox}

    The label data we loaded are integer values (1,2,3). We want human names
for the data classes we're working with.

    \begin{tcolorbox}[breakable, size=fbox, boxrule=1pt, pad at break*=1mm,colback=cellbackground, colframe=cellborder]
\prompt{In}{incolor}{27}{\boxspacing}
\begin{Verbatim}[commandchars=\\\{\}]
\PY{c+c1}{\PYZsh{} For humans:}
\PY{n}{class\PYZus{}names} \PY{o}{=} \PY{p}{[}\PY{l+s+s1}{\PYZsq{}}\PY{l+s+s1}{2x3 Brick}\PY{l+s+s1}{\PYZsq{}}\PY{p}{,} \PY{l+s+s1}{\PYZsq{}}\PY{l+s+s1}{2x2 Brick}\PY{l+s+s1}{\PYZsq{}}\PY{p}{,} \PY{l+s+s1}{\PYZsq{}}\PY{l+s+s1}{1x3 Brick}\PY{l+s+s1}{\PYZsq{}}\PY{p}{,} \PY{l+s+s1}{\PYZsq{}}\PY{l+s+s1}{2x1 Brick}\PY{l+s+s1}{\PYZsq{}}\PY{p}{,} \PY{l+s+s1}{\PYZsq{}}\PY{l+s+s1}{1x1 Brick}\PY{l+s+s1}{\PYZsq{}}\PY{p}{,} 
               \PY{l+s+s1}{\PYZsq{}}\PY{l+s+s1}{2x2 Macaroni}\PY{l+s+s1}{\PYZsq{}}\PY{p}{,} \PY{l+s+s1}{\PYZsq{}}\PY{l+s+s1}{2x2 Curved End}\PY{l+s+s1}{\PYZsq{}}\PY{p}{,} \PY{l+s+s1}{\PYZsq{}}\PY{l+s+s1}{Cog 16 Tooth}\PY{l+s+s1}{\PYZsq{}}\PY{p}{,} \PY{l+s+s1}{\PYZsq{}}\PY{l+s+s1}{1x2 Handles}\PY{l+s+s1}{\PYZsq{}}\PY{p}{,} \PY{l+s+s1}{\PYZsq{}}\PY{l+s+s1}{1x2 Grill}\PY{l+s+s1}{\PYZsq{}}\PY{p}{]}

\PY{c+c1}{\PYZsh{} Or the real LEGO codes:}
\PY{c+c1}{\PYZsh{} class\PYZus{}names = [\PYZsq{}3002\PYZsq{}, \PYZsq{}3003\PYZsq{}, \PYZsq{}3622\PYZsq{}, \PYZsq{}3004\PYZsq{}, \PYZsq{}3005\PYZsq{}, \PYZsq{}3063\PYZsq{}, \PYZsq{}47457\PYZsq{}, \PYZsq{}94925\PYZsq{}, \PYZsq{}3839a\PYZsq{}, \PYZsq{}2412b\PYZsq{}]}
\end{Verbatim}
\end{tcolorbox}

    \subsection{Convert to MXNet Tensors}\label{convert-to-mxnet-tensors}

    We have the data loaded into NDArrays. Now we transfer the data into
MXNet tensors.

Tensors act like arrays but with extra capability.

    \begin{tcolorbox}[breakable, size=fbox, boxrule=1pt, pad at break*=1mm,colback=cellbackground, colframe=cellborder]
\prompt{In}{incolor}{13}{\boxspacing}
\begin{Verbatim}[commandchars=\\\{\}]
\PY{n}{transformer} \PY{o}{=} \PY{n}{transforms}\PY{o}{.}\PY{n}{Compose}\PY{p}{(}\PY{p}{[}
    \PY{n}{transforms}\PY{o}{.}\PY{n}{ToTensor}\PY{p}{(}\PY{p}{)}\PY{p}{,}
    \PY{n}{transforms}\PY{o}{.}\PY{n}{Normalize}\PY{p}{(}\PY{l+m+mf}{0.13}\PY{p}{,} \PY{l+m+mf}{0.31}\PY{p}{)}\PY{p}{]}\PY{p}{)}

\PY{n}{train\PYZus{}data} \PY{o}{=} \PY{n}{train\PYZus{}data}\PY{o}{.}\PY{n}{transform\PYZus{}first}\PY{p}{(}\PY{n}{transformer}\PY{p}{)}
\PY{n}{test\PYZus{}data} \PY{o}{=} \PY{n}{test\PYZus{}data}\PY{o}{.}\PY{n}{transform\PYZus{}first}\PY{p}{(}\PY{n}{transformer}\PY{p}{)}
\end{Verbatim}
\end{tcolorbox}

    \begin{Verbatim}[commandchars=\\\{\}, frame=single, framerule=2mm, rulecolor=\color{outerrorbackground}]
\textcolor{ansi-red}{---------------------------------------------------------------------------}
\textcolor{ansi-red}{NameError}                                 Traceback (most recent call last)
Cell \textcolor{ansi-green}{In[13], line 5}
\textcolor{ansi-green-intense}{\textbf{      1}} transformer \def\tcRGB{\textcolor[RGB]}\expandafter\tcRGB\expandafter{\detokenize{98,98,98}}{=} transforms\def\tcRGB{\textcolor[RGB]}\expandafter\tcRGB\expandafter{\detokenize{98,98,98}}{.}Compose([
\textcolor{ansi-green-intense}{\textbf{      2}}     transforms\def\tcRGB{\textcolor[RGB]}\expandafter\tcRGB\expandafter{\detokenize{98,98,98}}{.}ToTensor(),
\textcolor{ansi-green-intense}{\textbf{      3}}     transforms\def\tcRGB{\textcolor[RGB]}\expandafter\tcRGB\expandafter{\detokenize{98,98,98}}{.}Normalize(\def\tcRGB{\textcolor[RGB]}\expandafter\tcRGB\expandafter{\detokenize{98,98,98}}{0.13}, \def\tcRGB{\textcolor[RGB]}\expandafter\tcRGB\expandafter{\detokenize{98,98,98}}{0.31})])
\textcolor{ansi-green}{----> 5} train\_data \def\tcRGB{\textcolor[RGB]}\expandafter\tcRGB\expandafter{\detokenize{98,98,98}}{=} \setlength{\fboxsep}{0pt}\colorbox{ansi-yellow}{train\_data\strut}\def\tcRGB{\textcolor[RGB]}\expandafter\tcRGB\expandafter{\detokenize{98,98,98}}{.}transform\_first(transformer)
\textcolor{ansi-green-intense}{\textbf{      6}} test\_data \def\tcRGB{\textcolor[RGB]}\expandafter\tcRGB\expandafter{\detokenize{98,98,98}}{=} test\_data\def\tcRGB{\textcolor[RGB]}\expandafter\tcRGB\expandafter{\detokenize{98,98,98}}{.}transform\_first(transformer)

\textcolor{ansi-red}{NameError}: name 'train\_data' is not defined
    \end{Verbatim}

    Let's take a look at one of the images loaded with the data.

    \begin{tcolorbox}[breakable, size=fbox, boxrule=1pt, pad at break*=1mm,colback=cellbackground, colframe=cellborder]
\prompt{In}{incolor}{25}{\boxspacing}
\begin{Verbatim}[commandchars=\\\{\}]
\PY{n}{train\PYZus{}image\PYZus{}no} \PY{o}{=} \PY{l+m+mi}{0}

\PY{n}{images\PYZus{}data}\PY{p}{,} \PY{n}{label\PYZus{}data} \PY{o}{=} \PY{n}{train\PYZus{}data}\PY{p}{[}\PY{n}{train\PYZus{}image\PYZus{}no}\PY{p}{]}
\PY{n}{plt}\PY{o}{.}\PY{n}{figure}\PY{p}{(}\PY{p}{)}
\PY{n}{plt}\PY{o}{.}\PY{n}{imshow}\PY{p}{(}\PY{n}{images\PYZus{}data}\PY{o}{.}\PY{n}{reshape}\PY{p}{(}\PY{p}{(}\PY{l+m+mi}{48}\PY{p}{,}\PY{l+m+mi}{48}\PY{p}{)}\PY{p}{)}\PY{o}{.}\PY{n}{asnumpy}\PY{p}{(}\PY{p}{)}\PY{p}{)}
\PY{n}{plt}\PY{o}{.}\PY{n}{colorbar}\PY{p}{(}\PY{p}{)}
\PY{n}{plt}\PY{o}{.}\PY{n}{xlabel}\PY{p}{(}\PY{n}{class\PYZus{}names}\PY{p}{[}\PY{n}{label\PYZus{}data}\PY{p}{]}\PY{p}{)}
\PY{n}{plt}\PY{o}{.}\PY{n}{show}\PY{p}{(}\PY{p}{)}
\end{Verbatim}
\end{tcolorbox}

    \begin{center}
    \adjustimage{max size={0.9\linewidth}{0.9\paperheight}}{output_24_0.png}
    \end{center}
    { \hspace*{\fill} \\}
    
    Let's look at some more of the data and make the formating a little
nicer.

    \begin{tcolorbox}[breakable, size=fbox, boxrule=1pt, pad at break*=1mm,colback=cellbackground, colframe=cellborder]
\prompt{In}{incolor}{24}{\boxspacing}
\begin{Verbatim}[commandchars=\\\{\}]
\PY{n}{plt}\PY{o}{.}\PY{n}{figure}\PY{p}{(}\PY{n}{figsize}\PY{o}{=}\PY{p}{(}\PY{l+m+mi}{15}\PY{p}{,}\PY{l+m+mi}{15}\PY{p}{)}\PY{p}{)}
\PY{k}{for} \PY{n}{i} \PY{o+ow}{in} \PY{n+nb}{range}\PY{p}{(}\PY{l+m+mi}{20}\PY{p}{)}\PY{p}{:}
    \PY{n}{images\PYZus{}data}\PY{p}{,} \PY{n}{label\PYZus{}data} \PY{o}{=} \PY{n}{train\PYZus{}data}\PY{p}{[}\PY{n}{i}\PY{p}{]}
    \PY{n}{plt}\PY{o}{.}\PY{n}{subplot}\PY{p}{(}\PY{l+m+mi}{5}\PY{p}{,}\PY{l+m+mi}{5}\PY{p}{,}\PY{n}{i}\PY{o}{+}\PY{l+m+mi}{1}\PY{p}{)}
    \PY{n}{plt}\PY{o}{.}\PY{n}{xticks}\PY{p}{(}\PY{p}{[}\PY{p}{]}\PY{p}{)}
    \PY{n}{plt}\PY{o}{.}\PY{n}{yticks}\PY{p}{(}\PY{p}{[}\PY{p}{]}\PY{p}{)}
    \PY{n}{plt}\PY{o}{.}\PY{n}{imshow}\PY{p}{(}\PY{n}{images\PYZus{}data}\PY{o}{.}\PY{n}{reshape}\PY{p}{(}\PY{p}{(}\PY{l+m+mi}{48}\PY{p}{,}\PY{l+m+mi}{48}\PY{p}{)}\PY{p}{)}\PY{o}{.}\PY{n}{asnumpy}\PY{p}{(}\PY{p}{)}\PY{p}{,} \PY{n}{cmap}\PY{o}{=}\PY{n}{plt}\PY{o}{.}\PY{n}{cm}\PY{o}{.}\PY{n}{binary}\PY{p}{)}
    \PY{n}{plt}\PY{o}{.}\PY{n}{xlabel}\PY{p}{(}\PY{n}{class\PYZus{}names}\PY{p}{[}\PY{n}{label\PYZus{}data}\PY{p}{]}\PY{p}{)}
\PY{n}{plt}\PY{o}{.}\PY{n}{show}\PY{p}{(}\PY{p}{)}
\end{Verbatim}
\end{tcolorbox}

    \begin{center}
    \adjustimage{max size={0.9\linewidth}{0.9\paperheight}}{output_26_0.png}
    \end{center}
    { \hspace*{\fill} \\}
    
    \section{3) Create the Model}\label{create-the-model}

    Now the data is loaded, so let's start training. First, we need to
create a model. With MXNet we often call the model object \texttt{net}.
We are creating an artificial neural network. It has 4 layers:

\begin{enumerate}
\def\labelenumi{\arabic{enumi}.}
\tightlist
\item
  The input layer with enough nodes for our image data.
\item
  A hidden layer with 128 nodes and ReLU activation.
\item
  A hidden layer with 64 nodes and ReLU activation.
\item
  An output layer with 10 nodes, one for each of the classes we want to
  identify.
\end{enumerate}

Each layer is densely connected, meaning that each neuron in one layer
is connected to every neuron in the next layer.

    \begin{tcolorbox}[breakable, size=fbox, boxrule=1pt, pad at break*=1mm,colback=cellbackground, colframe=cellborder]
\prompt{In}{incolor}{46}{\boxspacing}
\begin{Verbatim}[commandchars=\\\{\}]
\PY{c+c1}{\PYZsh{} TODO: Write the neural network model}
\PY{k+kn}{from} \PY{n+nn}{mxnet}\PY{n+nn}{.}\PY{n+nn}{gluon} \PY{k+kn}{import} \PY{n}{nn}

\PY{n}{net} \PY{o}{=} \PY{n}{nn}\PY{o}{.}\PY{n}{HybridSequential}\PY{p}{(}\PY{n}{prefix}\PY{o}{=}\PY{l+s+s1}{\PYZsq{}}\PY{l+s+s1}{MLP\PYZus{}}\PY{l+s+s1}{\PYZsq{}}\PY{p}{)}
\PY{k}{with} \PY{n}{net}\PY{o}{.}\PY{n}{name\PYZus{}scope}\PY{p}{(}\PY{p}{)}\PY{p}{:}
    \PY{n}{net}\PY{o}{.}\PY{n}{add}\PY{p}{(}
        \PY{n}{nn}\PY{o}{.}\PY{n}{Flatten}\PY{p}{(}\PY{p}{)}\PY{p}{,}
        \PY{n}{nn}\PY{o}{.}\PY{n}{Dense}\PY{p}{(}\PY{l+m+mi}{256}\PY{p}{,} \PY{n}{activation}\PY{o}{=}\PY{l+s+s1}{\PYZsq{}}\PY{l+s+s1}{relu}\PY{l+s+s1}{\PYZsq{}}\PY{p}{)}\PY{p}{,}  \PY{c+c1}{\PYZsh{} New additional layer}
        \PY{n}{nn}\PY{o}{.}\PY{n}{Dense}\PY{p}{(}\PY{l+m+mi}{128}\PY{p}{,} \PY{n}{activation}\PY{o}{=}\PY{l+s+s1}{\PYZsq{}}\PY{l+s+s1}{relu}\PY{l+s+s1}{\PYZsq{}}\PY{p}{)}\PY{p}{,}
        \PY{n}{nn}\PY{o}{.}\PY{n}{Dense}\PY{p}{(}\PY{l+m+mi}{64}\PY{p}{,} \PY{n}{activation}\PY{o}{=}\PY{l+s+s1}{\PYZsq{}}\PY{l+s+s1}{relu}\PY{l+s+s1}{\PYZsq{}}\PY{p}{)}\PY{p}{,}
        \PY{n}{nn}\PY{o}{.}\PY{n}{Dense}\PY{p}{(}\PY{l+m+mi}{32}\PY{p}{,} \PY{n}{activation}\PY{o}{=}\PY{l+s+s1}{\PYZsq{}}\PY{l+s+s1}{relu}\PY{l+s+s1}{\PYZsq{}}\PY{p}{)}\PY{p}{,}  \PY{c+c1}{\PYZsh{} New additional layer}
        \PY{n}{nn}\PY{o}{.}\PY{n}{Dense}\PY{p}{(}\PY{l+m+mi}{10}\PY{p}{,} \PY{n}{activation}\PY{o}{=}\PY{k+kc}{None}\PY{p}{)}
    \PY{p}{)}

\PY{c+c1}{\PYZsh{} Initialize the model}
\PY{n}{net}\PY{o}{.}\PY{n}{initialize}\PY{p}{(}\PY{p}{)}

\PY{c+c1}{\PYZsh{} Print the network architecture}
\PY{n+nb}{print}\PY{p}{(}\PY{n}{net}\PY{p}{)}
\end{Verbatim}
\end{tcolorbox}

    \begin{Verbatim}[commandchars=\\\{\}]
HybridSequential(
  (0): Flatten
  (1): Dense(None -> 256, Activation(relu))
  (2): Dense(None -> 128, Activation(relu))
  (3): Dense(None -> 64, Activation(relu))
  (4): Dense(None -> 32, Activation(relu))
  (5): Dense(None -> 10, linear)
)
    \end{Verbatim}

    MXNet Gluon provides data loaders we can use to simplify the loading of
data when training our model. Let's set them up.

    \begin{tcolorbox}[breakable, size=fbox, boxrule=1pt, pad at break*=1mm,colback=cellbackground, colframe=cellborder]
\prompt{In}{incolor}{47}{\boxspacing}
\begin{Verbatim}[commandchars=\\\{\}]
\PY{n}{batch\PYZus{}size} \PY{o}{=} \PY{l+m+mi}{34}
\PY{n}{train\PYZus{}loader} \PY{o}{=} \PY{n}{mx}\PY{o}{.}\PY{n}{gluon}\PY{o}{.}\PY{n}{data}\PY{o}{.}\PY{n}{DataLoader}\PY{p}{(}\PY{n}{train\PYZus{}data}\PY{p}{,} \PY{n}{shuffle}\PY{o}{=}\PY{k+kc}{True}\PY{p}{,} \PY{n}{batch\PYZus{}size}\PY{o}{=}\PY{n}{batch\PYZus{}size}\PY{p}{)}
\end{Verbatim}
\end{tcolorbox}

    Initializing the model. Note we pass in the variable holding the
processor type here.

    \begin{tcolorbox}[breakable, size=fbox, boxrule=1pt, pad at break*=1mm,colback=cellbackground, colframe=cellborder]
\prompt{In}{incolor}{48}{\boxspacing}
\begin{Verbatim}[commandchars=\\\{\}]
\PY{n}{net}\PY{o}{.}\PY{n}{initialize}\PY{p}{(}\PY{n}{mx}\PY{o}{.}\PY{n}{init}\PY{o}{.}\PY{n}{Xavier}\PY{p}{(}\PY{p}{)}\PY{p}{,} \PY{n}{ctx}\PY{o}{=}\PY{n}{ctx}\PY{p}{)}
\end{Verbatim}
\end{tcolorbox}

    \begin{Verbatim}[commandchars=\\\{\}]
/home/ec2-user/anaconda3/envs/python3/lib/python3.10/site-
packages/mxnet/gluon/parameter.py:896: UserWarning: Parameter 'MLP\_dense0\_bias'
is already initialized, ignoring. Set force\_reinit=True to re-initialize.
  v.initialize(None, ctx, init, force\_reinit=force\_reinit)
/home/ec2-user/anaconda3/envs/python3/lib/python3.10/site-
packages/mxnet/gluon/parameter.py:896: UserWarning: Parameter 'MLP\_dense1\_bias'
is already initialized, ignoring. Set force\_reinit=True to re-initialize.
  v.initialize(None, ctx, init, force\_reinit=force\_reinit)
/home/ec2-user/anaconda3/envs/python3/lib/python3.10/site-
packages/mxnet/gluon/parameter.py:896: UserWarning: Parameter 'MLP\_dense2\_bias'
is already initialized, ignoring. Set force\_reinit=True to re-initialize.
  v.initialize(None, ctx, init, force\_reinit=force\_reinit)
/home/ec2-user/anaconda3/envs/python3/lib/python3.10/site-
packages/mxnet/gluon/parameter.py:896: UserWarning: Parameter 'MLP\_dense3\_bias'
is already initialized, ignoring. Set force\_reinit=True to re-initialize.
  v.initialize(None, ctx, init, force\_reinit=force\_reinit)
/home/ec2-user/anaconda3/envs/python3/lib/python3.10/site-
packages/mxnet/gluon/parameter.py:896: UserWarning: Parameter 'MLP\_dense4\_bias'
is already initialized, ignoring. Set force\_reinit=True to re-initialize.
  v.initialize(None, ctx, init, force\_reinit=force\_reinit)
    \end{Verbatim}

    Gluon provides a trainer object to maintain the state of the training.
We create it here, and use it in the training process.

    \begin{tcolorbox}[breakable, size=fbox, boxrule=1pt, pad at break*=1mm,colback=cellbackground, colframe=cellborder]
\prompt{In}{incolor}{49}{\boxspacing}
\begin{Verbatim}[commandchars=\\\{\}]
\PY{n}{trainer} \PY{o}{=} \PY{n}{gluon}\PY{o}{.}\PY{n}{Trainer}\PY{p}{(}
    \PY{n}{params}\PY{o}{=}\PY{n}{net}\PY{o}{.}\PY{n}{collect\PYZus{}params}\PY{p}{(}\PY{p}{)}\PY{p}{,}
    \PY{n}{optimizer}\PY{o}{=}\PY{l+s+s1}{\PYZsq{}}\PY{l+s+s1}{sgd}\PY{l+s+s1}{\PYZsq{}}\PY{p}{,}
    \PY{n}{optimizer\PYZus{}params}\PY{o}{=}\PY{p}{\PYZob{}}\PY{l+s+s1}{\PYZsq{}}\PY{l+s+s1}{learning\PYZus{}rate}\PY{l+s+s1}{\PYZsq{}}\PY{p}{:} \PY{l+m+mf}{0.04}\PY{p}{\PYZcb{}}\PY{p}{,}
\PY{p}{)}
\end{Verbatim}
\end{tcolorbox}

    We're almost ready to train. First, we define the metric to use while we
train and the loss function to use. Gluon provides a softmax loss
function, so we just use that.

    \begin{tcolorbox}[breakable, size=fbox, boxrule=1pt, pad at break*=1mm,colback=cellbackground, colframe=cellborder]
\prompt{In}{incolor}{50}{\boxspacing}
\begin{Verbatim}[commandchars=\\\{\}]
\PY{n}{metric} \PY{o}{=} \PY{n}{mx}\PY{o}{.}\PY{n}{metric}\PY{o}{.}\PY{n}{Accuracy}\PY{p}{(}\PY{p}{)}
\PY{n}{loss\PYZus{}function} \PY{o}{=} \PY{n}{gluon}\PY{o}{.}\PY{n}{loss}\PY{o}{.}\PY{n}{SoftmaxCrossEntropyLoss}\PY{p}{(}\PY{p}{)}
\end{Verbatim}
\end{tcolorbox}

    Now we train. We could write the following code into a \texttt{fit}
function, but this inline code does the job.

    \begin{tcolorbox}[breakable, size=fbox, boxrule=1pt, pad at break*=1mm,colback=cellbackground, colframe=cellborder]
\prompt{In}{incolor}{51}{\boxspacing}
\begin{Verbatim}[commandchars=\\\{\}]
\PY{n}{num\PYZus{}epochs} \PY{o}{=} \PY{l+m+mi}{10}
\PY{n}{history} \PY{o}{=} \PY{p}{[}\PY{p}{]}
    
\PY{k}{for} \PY{n}{epoch} \PY{o+ow}{in} \PY{n+nb}{range}\PY{p}{(}\PY{n}{num\PYZus{}epochs}\PY{p}{)}\PY{p}{:}
    \PY{k}{for} \PY{n}{inputs}\PY{p}{,} \PY{n}{labels} \PY{o+ow}{in} \PY{n}{train\PYZus{}loader}\PY{p}{:}
        \PY{c+c1}{\PYZsh{} Possibly copy inputs and labels to the CPU}
        \PY{n}{inputs} \PY{o}{=} \PY{n}{inputs}\PY{o}{.}\PY{n}{as\PYZus{}in\PYZus{}context}\PY{p}{(}\PY{n}{ctx}\PY{p}{)}
        \PY{n}{labels} \PY{o}{=} \PY{n}{labels}\PY{o}{.}\PY{n}{as\PYZus{}in\PYZus{}context}\PY{p}{(}\PY{n}{ctx}\PY{p}{)}

        \PY{c+c1}{\PYZsh{} Forward pass}
        \PY{k}{with} \PY{n}{autograd}\PY{o}{.}\PY{n}{record}\PY{p}{(}\PY{p}{)}\PY{p}{:}
            \PY{n}{outputs} \PY{o}{=} \PY{n}{net}\PY{p}{(}\PY{n}{inputs}\PY{p}{)}
            \PY{n}{loss} \PY{o}{=} \PY{n}{loss\PYZus{}function}\PY{p}{(}\PY{n}{outputs}\PY{p}{,} \PY{n}{labels}\PY{p}{)}

        \PY{c+c1}{\PYZsh{} Backpropagation}
        \PY{n}{loss}\PY{o}{.}\PY{n}{backward}\PY{p}{(}\PY{p}{)}
        \PY{n}{metric}\PY{o}{.}\PY{n}{update}\PY{p}{(}\PY{n}{labels}\PY{p}{,} \PY{n}{outputs}\PY{p}{)}

        \PY{c+c1}{\PYZsh{} Update}
        \PY{n}{trainer}\PY{o}{.}\PY{n}{step}\PY{p}{(}\PY{n}{batch\PYZus{}size}\PY{o}{=}\PY{n}{inputs}\PY{o}{.}\PY{n}{shape}\PY{p}{[}\PY{l+m+mi}{0}\PY{p}{]}\PY{p}{)}

    \PY{c+c1}{\PYZsh{} Print the evaluation metric and reset it for the next epoch}
    \PY{n}{name}\PY{p}{,} \PY{n}{acc} \PY{o}{=} \PY{n}{metric}\PY{o}{.}\PY{n}{get}\PY{p}{(}\PY{p}{)}
    \PY{n}{history}\PY{o}{.}\PY{n}{insert}\PY{p}{(}\PY{n}{epoch}\PY{p}{,}\PY{n}{acc}\PY{p}{)}
    \PY{n+nb}{print}\PY{p}{(}\PY{l+s+s1}{\PYZsq{}}\PY{l+s+s1}{.}\PY{l+s+s1}{\PYZsq{}}\PY{p}{,} \PY{n}{end}\PY{o}{=}\PY{l+s+s1}{\PYZsq{}}\PY{l+s+s1}{\PYZsq{}}\PY{p}{)}
    \PY{n}{metric}\PY{o}{.}\PY{n}{reset}\PY{p}{(}\PY{p}{)}

\PY{n+nb}{print}\PY{p}{(}\PY{l+s+s1}{\PYZsq{}}\PY{l+s+s1}{[Done]}\PY{l+s+s1}{\PYZsq{}}\PY{p}{)}
\end{Verbatim}
\end{tcolorbox}

    \begin{Verbatim}[commandchars=\\\{\}]
{\ldots}[Done]
    \end{Verbatim}

    \section{4) Evaluate the Model}\label{evaluate-the-model}

    During the training loop we collected accuracy data in each epoch. Let's
graph this data to get a sense of how the training went.

    \begin{tcolorbox}[breakable, size=fbox, boxrule=1pt, pad at break*=1mm,colback=cellbackground, colframe=cellborder]
\prompt{In}{incolor}{52}{\boxspacing}
\begin{Verbatim}[commandchars=\\\{\}]
\PY{n}{plt}\PY{o}{.}\PY{n}{figure}\PY{p}{(}\PY{n}{figsize}\PY{o}{=}\PY{p}{(}\PY{l+m+mi}{7}\PY{p}{,} \PY{l+m+mi}{4}\PY{p}{)}\PY{p}{)}
\PY{n}{plt}\PY{o}{.}\PY{n}{plot}\PY{p}{(}\PY{n}{history}\PY{p}{)}
\PY{n}{plt}\PY{o}{.}\PY{n}{title}\PY{p}{(}\PY{l+s+s1}{\PYZsq{}}\PY{l+s+s1}{Model accuracy}\PY{l+s+s1}{\PYZsq{}}\PY{p}{)}
\PY{n}{plt}\PY{o}{.}\PY{n}{ylabel}\PY{p}{(}\PY{l+s+s1}{\PYZsq{}}\PY{l+s+s1}{Accuracy}\PY{l+s+s1}{\PYZsq{}}\PY{p}{)}
\PY{n}{plt}\PY{o}{.}\PY{n}{xlabel}\PY{p}{(}\PY{l+s+s1}{\PYZsq{}}\PY{l+s+s1}{Epoch}\PY{l+s+s1}{\PYZsq{}}\PY{p}{)}
\end{Verbatim}
\end{tcolorbox}

            \begin{tcolorbox}[breakable, size=fbox, boxrule=.5pt, pad at break*=1mm, opacityfill=0]
\prompt{Out}{outcolor}{52}{\boxspacing}
\begin{Verbatim}[commandchars=\\\{\}]
Text(0.5, 0, 'Epoch')
\end{Verbatim}
\end{tcolorbox}
        
    \begin{center}
    \adjustimage{max size={0.9\linewidth}{0.9\paperheight}}{output_42_1.png}
    \end{center}
    { \hspace*{\fill} \\}
    
    Now, we use the test data to perform an accuracy measurement. Is the
accuracy with the testing data much lower than the end of training? If
so our model might be overfit.

We use the Gluon data loader again, this time with the test data.

    \begin{tcolorbox}[breakable, size=fbox, boxrule=1pt, pad at break*=1mm,colback=cellbackground, colframe=cellborder]
\prompt{In}{incolor}{100}{\boxspacing}
\begin{Verbatim}[commandchars=\\\{\}]
\PY{c+c1}{\PYZsh{} TODO: Build our `test\PYZus{}loader` using Gluon\PYZsq{}s Data Loader with the `test\PYZus{}data`}
\PY{n}{test\PYZus{}loader} \PY{o}{=} \PY{n}{mx}\PY{o}{.}\PY{n}{gluon}\PY{o}{.}\PY{n}{data}\PY{o}{.}\PY{n}{DataLoader}\PY{p}{(}\PY{n}{test\PYZus{}data}\PY{p}{,} \PY{n}{shuffle}\PY{o}{=}\PY{k+kc}{False}\PY{p}{,}
\PY{n}{batch\PYZus{}size}\PY{o}{=}\PY{n}{batch\PYZus{}size}\PY{p}{)}
\end{Verbatim}
\end{tcolorbox}

    And measure the accuracy.

    \begin{tcolorbox}[breakable, size=fbox, boxrule=1pt, pad at break*=1mm,colback=cellbackground, colframe=cellborder]
\prompt{In}{incolor}{101}{\boxspacing}
\begin{Verbatim}[commandchars=\\\{\}]
\PY{n}{metric} \PY{o}{=} \PY{n}{mx}\PY{o}{.}\PY{n}{metric}\PY{o}{.}\PY{n}{Accuracy}\PY{p}{(}\PY{p}{)}
\PY{k}{for} \PY{n}{inputs}\PY{p}{,} \PY{n}{labels} \PY{o+ow}{in} \PY{n}{test\PYZus{}loader}\PY{p}{:}
    \PY{n}{inputs} \PY{o}{=} \PY{n}{inputs}\PY{o}{.}\PY{n}{as\PYZus{}in\PYZus{}context}\PY{p}{(}\PY{n}{ctx}\PY{p}{)}
    \PY{n}{labels} \PY{o}{=} \PY{n}{labels}\PY{o}{.}\PY{n}{as\PYZus{}in\PYZus{}context}\PY{p}{(}\PY{n}{ctx}\PY{p}{)}
    \PY{n}{metric}\PY{o}{.}\PY{n}{update}\PY{p}{(}\PY{n}{labels}\PY{p}{,} \PY{n}{net}\PY{p}{(}\PY{n}{inputs}\PY{p}{)}\PY{p}{)}
\PY{n}{metric\PYZus{}name}\PY{p}{,} \PY{n}{metric\PYZus{}value} \PY{o}{=} \PY{n}{metric}\PY{o}{.}\PY{n}{get}\PY{p}{(}\PY{p}{)}
\PY{n+nb}{print}\PY{p}{(}\PY{l+s+sa}{f}\PY{l+s+s1}{\PYZsq{}}\PY{l+s+s1}{Validation: }\PY{l+s+si}{\PYZob{}}\PY{n}{metric\PYZus{}name}\PY{l+s+si}{\PYZcb{}}\PY{l+s+s1}{ = }\PY{l+s+si}{\PYZob{}}\PY{n}{metric\PYZus{}value}\PY{l+s+si}{\PYZcb{}}\PY{l+s+s1}{\PYZsq{}}\PY{p}{)}
\end{Verbatim}
\end{tcolorbox}

    \begin{Verbatim}[commandchars=\\\{\}]
Validation: accuracy = 0.9533333333333334
    \end{Verbatim}

    \section{5) Test the Model}\label{test-the-model}

In order to make our tests look good, we define a couple of functions to
dispaly the results.

    \begin{tcolorbox}[breakable, size=fbox, boxrule=1pt, pad at break*=1mm,colback=cellbackground, colframe=cellborder]
\prompt{In}{incolor}{55}{\boxspacing}
\begin{Verbatim}[commandchars=\\\{\}]
\PY{c+c1}{\PYZsh{} Function to display the image:}
\PY{k}{def} \PY{n+nf}{plot\PYZus{}image}\PY{p}{(}\PY{n}{predictions\PYZus{}array}\PY{p}{,} \PY{n}{true\PYZus{}label}\PY{p}{,} \PY{n}{img}\PY{p}{)}\PY{p}{:}
    \PY{n}{plt}\PY{o}{.}\PY{n}{xticks}\PY{p}{(}\PY{p}{[}\PY{p}{]}\PY{p}{)}
    \PY{n}{plt}\PY{o}{.}\PY{n}{yticks}\PY{p}{(}\PY{p}{[}\PY{p}{]}\PY{p}{)}
    \PY{c+c1}{\PYZsh{}Image data is currently a flat array. Convert to matrix for display}
    \PY{n}{plt}\PY{o}{.}\PY{n}{imshow}\PY{p}{(}\PY{n}{img}\PY{o}{.}\PY{n}{reshape}\PY{p}{(}\PY{p}{(}\PY{l+m+mi}{48}\PY{p}{,}\PY{l+m+mi}{48}\PY{p}{)}\PY{p}{)}\PY{o}{.}\PY{n}{asnumpy}\PY{p}{(}\PY{p}{)}\PY{p}{,} \PY{n}{cmap}\PY{o}{=}\PY{n}{plt}\PY{o}{.}\PY{n}{cm}\PY{o}{.}\PY{n}{binary}\PY{p}{)}
    \PY{n}{predicted\PYZus{}label} \PY{o}{=} \PY{n}{np}\PY{o}{.}\PY{n}{argmax}\PY{p}{(}\PY{n}{predictions\PYZus{}array}\PY{p}{)}
    \PY{n}{probability} \PY{o}{=} \PY{l+m+mi}{100} \PY{o}{*} \PY{n}{np}\PY{o}{.}\PY{n}{max}\PY{p}{(}\PY{n}{predictions\PYZus{}array}\PY{p}{)}
    \PY{n}{color} \PY{o}{=} \PY{l+s+s1}{\PYZsq{}}\PY{l+s+s1}{green}\PY{l+s+s1}{\PYZsq{}} \PY{k}{if} \PY{n}{predicted\PYZus{}label} \PY{o}{==} \PY{n}{true\PYZus{}label} \PY{k}{else} \PY{l+s+s1}{\PYZsq{}}\PY{l+s+s1}{red}\PY{l+s+s1}{\PYZsq{}}
    \PY{c+c1}{\PYZsh{} Print a label with \PYZsq{}predicted class\PYZsq{} \PYZsq{}actual class\PYZsq{}}
    \PY{n}{plt}\PY{o}{.}\PY{n}{xlabel}\PY{p}{(}\PY{l+s+sa}{f}\PY{l+s+s2}{\PYZdq{}}\PY{l+s+si}{\PYZob{}}\PY{n}{class\PYZus{}names}\PY{p}{[}\PY{n}{predicted\PYZus{}label}\PY{p}{]}\PY{l+s+si}{\PYZcb{}}\PY{l+s+s2}{ (}\PY{l+s+si}{\PYZob{}}\PY{n}{true\PYZus{}label}\PY{l+s+si}{\PYZcb{}}\PY{l+s+s2}{ \PYZhy{} }\PY{l+s+si}{\PYZob{}}\PY{n}{class\PYZus{}names}\PY{p}{[}\PY{n}{true\PYZus{}label}\PY{p}{]}\PY{l+s+si}{\PYZcb{}}\PY{l+s+s2}{)}\PY{l+s+s2}{\PYZdq{}}\PY{p}{,}
                                \PY{n}{color}\PY{o}{=}\PY{n}{color}\PY{p}{)}

\PY{c+c1}{\PYZsh{} Function to display the prediction results in a graph:}
\PY{k}{def} \PY{n+nf}{plot\PYZus{}value\PYZus{}array}\PY{p}{(}\PY{n}{predictions\PYZus{}array}\PY{p}{,} \PY{n}{true\PYZus{}label}\PY{p}{)}\PY{p}{:}
    \PY{n}{plt}\PY{o}{.}\PY{n}{xticks}\PY{p}{(}\PY{n+nb}{range}\PY{p}{(}\PY{l+m+mi}{10}\PY{p}{)}\PY{p}{)}
    \PY{n}{plt}\PY{o}{.}\PY{n}{yticks}\PY{p}{(}\PY{p}{[}\PY{p}{]}\PY{p}{)}
    \PY{n}{plot} \PY{o}{=} \PY{n}{plt}\PY{o}{.}\PY{n}{bar}\PY{p}{(}\PY{n+nb}{range}\PY{p}{(}\PY{l+m+mi}{10}\PY{p}{)}\PY{p}{,} \PY{n}{predictions\PYZus{}array}\PY{p}{,} \PY{n}{color}\PY{o}{=}\PY{l+s+s2}{\PYZdq{}}\PY{l+s+s2}{\PYZsh{}777777}\PY{l+s+s2}{\PYZdq{}}\PY{p}{)}
    \PY{n}{predicted\PYZus{}label} \PY{o}{=} \PY{n}{np}\PY{o}{.}\PY{n}{argmax}\PY{p}{(}\PY{n}{predictions\PYZus{}array}\PY{p}{)}
    \PY{n}{plot}\PY{p}{[}\PY{n}{predicted\PYZus{}label}\PY{p}{]}\PY{o}{.}\PY{n}{set\PYZus{}color}\PY{p}{(}\PY{l+s+s1}{\PYZsq{}}\PY{l+s+s1}{red}\PY{l+s+s1}{\PYZsq{}}\PY{p}{)}
    \PY{n}{plot}\PY{p}{[}\PY{n}{true\PYZus{}label}\PY{p}{]}\PY{o}{.}\PY{n}{set\PYZus{}color}\PY{p}{(}\PY{l+s+s1}{\PYZsq{}}\PY{l+s+s1}{green}\PY{l+s+s1}{\PYZsq{}}\PY{p}{)}
\end{Verbatim}
\end{tcolorbox}

    \subsection{Single Prediction}\label{single-prediction}

Let's test out model. Choose one of the images from our test set.

    \begin{tcolorbox}[breakable, size=fbox, boxrule=1pt, pad at break*=1mm,colback=cellbackground, colframe=cellborder]
\prompt{In}{incolor}{90}{\boxspacing}
\begin{Verbatim}[commandchars=\\\{\}]
\PY{n}{prediction\PYZus{}image\PYZus{}number} \PY{o}{=} \PY{l+m+mi}{25}
\PY{n}{prediction\PYZus{}image}\PY{p}{,} \PY{n}{prediction\PYZus{}label} \PY{o}{=} \PY{n}{test\PYZus{}data}\PY{p}{[}\PY{n}{prediction\PYZus{}image\PYZus{}number}\PY{p}{]}
\end{Verbatim}
\end{tcolorbox}

    Now make a prediction.

    \begin{tcolorbox}[breakable, size=fbox, boxrule=1pt, pad at break*=1mm,colback=cellbackground, colframe=cellborder]
\prompt{In}{incolor}{91}{\boxspacing}
\begin{Verbatim}[commandchars=\\\{\}]
\PY{c+c1}{\PYZsh{} TODO: Run the `prediction\PYZus{}image` through the model}
\PY{n}{predictions\PYZus{}single} \PY{o}{=} \PY{n}{net}\PY{p}{(}\PY{n}{prediction\PYZus{}image}\PY{p}{)}
\PY{n}{predictions\PYZus{}single}
\end{Verbatim}
\end{tcolorbox}

            \begin{tcolorbox}[breakable, size=fbox, boxrule=.5pt, pad at break*=1mm, opacityfill=0]
\prompt{Out}{outcolor}{91}{\boxspacing}
\begin{Verbatim}[commandchars=\\\{\}]

[[-0.8601913 -1.2062907  8.0775    -2.9179156 -1.4651215  4.3540826
  -0.6141164  2.149591  -0.7891287 -3.9755204]]
<NDArray 1x10 @cpu(0)>
\end{Verbatim}
\end{tcolorbox}
        
    Let's plot a bar chart using the helper function. This gives us a sense
of how well the model classified this image. The predicted label will be
red if it is different than the actual label. The actual label will be
green.

    \begin{tcolorbox}[breakable, size=fbox, boxrule=1pt, pad at break*=1mm,colback=cellbackground, colframe=cellborder]
\prompt{In}{incolor}{92}{\boxspacing}
\begin{Verbatim}[commandchars=\\\{\}]
\PY{n}{plot\PYZus{}value\PYZus{}array}\PY{p}{(}\PY{n}{predictions\PYZus{}single}\PY{p}{[}\PY{l+m+mi}{0}\PY{p}{]}\PY{o}{.}\PY{n}{asnumpy}\PY{p}{(}\PY{p}{)}\PY{p}{,} \PY{n}{prediction\PYZus{}label}\PY{p}{)}
\PY{n}{plt}\PY{o}{.}\PY{n}{xticks}\PY{p}{(}\PY{n+nb}{range}\PY{p}{(}\PY{l+m+mi}{10}\PY{p}{)}\PY{p}{,} \PY{n}{class\PYZus{}names}\PY{p}{,} \PY{n}{rotation}\PY{o}{=}\PY{l+m+mi}{45}\PY{p}{)}
\PY{n}{plt}\PY{o}{.}\PY{n}{show}\PY{p}{(}\PY{p}{)}
\end{Verbatim}
\end{tcolorbox}

    \begin{center}
    \adjustimage{max size={0.9\linewidth}{0.9\paperheight}}{output_54_0.png}
    \end{center}
    { \hspace*{\fill} \\}
    
    And which block should we have found? In other words, did we get it
right?

    \begin{tcolorbox}[breakable, size=fbox, boxrule=1pt, pad at break*=1mm,colback=cellbackground, colframe=cellborder]
\prompt{In}{incolor}{93}{\boxspacing}
\begin{Verbatim}[commandchars=\\\{\}]
\PY{n}{plot\PYZus{}image}\PY{p}{(}\PY{n}{predictions\PYZus{}single}\PY{p}{[}\PY{l+m+mi}{0}\PY{p}{]}\PY{o}{.}\PY{n}{asnumpy}\PY{p}{(}\PY{p}{)}\PY{p}{,} \PY{n}{prediction\PYZus{}label}\PY{p}{,} \PY{n}{prediction\PYZus{}image}\PY{p}{)}
\end{Verbatim}
\end{tcolorbox}

    \begin{center}
    \adjustimage{max size={0.9\linewidth}{0.9\paperheight}}{output_56_0.png}
    \end{center}
    { \hspace*{\fill} \\}
    
    \subsection{Batch Prediction}\label{batch-prediction}

Now lets get prediction values for \textbf{all} the test images we have.
This time we don't use a data loader, we just iterate through the raw
test image data.

    \begin{tcolorbox}[breakable, size=fbox, boxrule=1pt, pad at break*=1mm,colback=cellbackground, colframe=cellborder]
\prompt{In}{incolor}{97}{\boxspacing}
\begin{Verbatim}[commandchars=\\\{\}]
\PY{n}{predictions} \PY{o}{=} \PY{p}{[}\PY{p}{]}
\PY{n}{test\PYZus{}labels} \PY{o}{=} \PY{p}{[}\PY{p}{]}

\PY{k}{for} \PY{n}{i} \PY{o+ow}{in} \PY{n}{test\PYZus{}data}\PY{p}{:}
    \PY{n}{pred\PYZus{}image}\PY{p}{,} \PY{n}{pred\PYZus{}label} \PY{o}{=} \PY{n}{i}
    \PY{n}{p} \PY{o}{=} \PY{n}{net}\PY{p}{(}\PY{n}{pred\PYZus{}image}\PY{p}{)}
    \PY{n}{predictions}\PY{o}{.}\PY{n}{append}\PY{p}{(}\PY{n}{p}\PY{p}{)}
    \PY{n}{test\PYZus{}labels}\PY{o}{.}\PY{n}{append}\PY{p}{(}\PY{n}{pred\PYZus{}label}\PY{p}{)}    
\end{Verbatim}
\end{tcolorbox}

    Finally, let's use our helper functions to summarize the first 16 images
in our test data. How did we do?

    \begin{tcolorbox}[breakable, size=fbox, boxrule=1pt, pad at break*=1mm,colback=cellbackground, colframe=cellborder]
\prompt{In}{incolor}{99}{\boxspacing}
\begin{Verbatim}[commandchars=\\\{\}]
\PY{n}{num\PYZus{}rows} \PY{o}{=} \PY{l+m+mi}{8}
\PY{n}{num\PYZus{}cols} \PY{o}{=} \PY{l+m+mi}{2}
\PY{n}{num\PYZus{}images} \PY{o}{=} \PY{n}{num\PYZus{}rows}\PY{o}{*}\PY{n}{num\PYZus{}cols}
\PY{n}{plt}\PY{o}{.}\PY{n}{figure}\PY{p}{(}\PY{n}{figsize}\PY{o}{=}\PY{p}{(}\PY{l+m+mi}{15}\PY{p}{,} \PY{l+m+mi}{16}\PY{p}{)}\PY{p}{)}
\PY{k}{for} \PY{n}{i} \PY{o+ow}{in} \PY{n+nb}{range}\PY{p}{(}\PY{n}{num\PYZus{}images}\PY{p}{)}\PY{p}{:}
    \PY{n}{plt}\PY{o}{.}\PY{n}{subplot}\PY{p}{(}\PY{n}{num\PYZus{}rows}\PY{p}{,} \PY{l+m+mi}{2}\PY{o}{*}\PY{n}{num\PYZus{}cols}\PY{p}{,} \PY{l+m+mi}{2}\PY{o}{*}\PY{n}{i}\PY{o}{+}\PY{l+m+mi}{1}\PY{p}{)}
    \PY{n}{plot\PYZus{}image}\PY{p}{(}\PY{n}{predictions}\PY{p}{[}\PY{n}{i}\PY{p}{]}\PY{o}{.}\PY{n}{asnumpy}\PY{p}{(}\PY{p}{)}\PY{p}{,} \PY{n}{test\PYZus{}data}\PY{p}{[}\PY{n}{i}\PY{p}{]}\PY{p}{[}\PY{l+m+mi}{1}\PY{p}{]}\PY{p}{,} \PY{n}{test\PYZus{}data}\PY{p}{[}\PY{n}{i}\PY{p}{]}\PY{p}{[}\PY{l+m+mi}{0}\PY{p}{]}\PY{p}{)}
    \PY{n}{plt}\PY{o}{.}\PY{n}{subplot}\PY{p}{(}\PY{n}{num\PYZus{}rows}\PY{p}{,} \PY{l+m+mi}{2}\PY{o}{*}\PY{n}{num\PYZus{}cols}\PY{p}{,} \PY{l+m+mi}{2}\PY{o}{*}\PY{n}{i}\PY{o}{+}\PY{l+m+mi}{2}\PY{p}{)}
    \PY{n}{plot\PYZus{}value\PYZus{}array}\PY{p}{(}\PY{n}{predictions}\PY{p}{[}\PY{n}{i}\PY{p}{]}\PY{p}{[}\PY{l+m+mi}{0}\PY{p}{]}\PY{o}{.}\PY{n}{asnumpy}\PY{p}{(}\PY{p}{)}\PY{p}{,} \PY{n}{test\PYZus{}data}\PY{p}{[}\PY{n}{i}\PY{p}{]}\PY{p}{[}\PY{l+m+mi}{1}\PY{p}{]}\PY{p}{)}
\PY{n}{plt}\PY{o}{.}\PY{n}{show}\PY{p}{(}\PY{p}{)}
\end{Verbatim}
\end{tcolorbox}

    \begin{center}
    \adjustimage{max size={0.9\linewidth}{0.9\paperheight}}{output_60_0.png}
    \end{center}
    { \hspace*{\fill} \\}
    
    \subsection{Conclusion}\label{conclusion}

I got an accuracy of about 95\%. Your model might be slightly higher or
lower. For your actual problem (sorting the blocks), is approximately
95\% accuracy good enough for you? Are you willing to have about 1 in 20
of your bricks in the wrong place for the advantage of not having to
manually decide which type of brick you are sorting? That's up to you!
You can also analyze which bricks the model is consistently getting
wrong and reorganize those bins yourself once the model is finished.
Machine learning is just one tool and one option.

Use the rest of your lab time to experiment with the model architecture
to see if you can improve on your current accuracy. Try using another
random seed. You can also change the number of training epochs, but be
careful not to overfit your model!

    \begin{tcolorbox}[breakable, size=fbox, boxrule=1pt, pad at break*=1mm,colback=cellbackground, colframe=cellborder]
\prompt{In}{incolor}{ }{\boxspacing}
\begin{Verbatim}[commandchars=\\\{\}]

\end{Verbatim}
\end{tcolorbox}

    \begin{tcolorbox}[breakable, size=fbox, boxrule=1pt, pad at break*=1mm,colback=cellbackground, colframe=cellborder]
\prompt{In}{incolor}{ }{\boxspacing}
\begin{Verbatim}[commandchars=\\\{\}]

\end{Verbatim}
\end{tcolorbox}


    % Add a bibliography block to the postdoc
    
    
    
\end{document}
